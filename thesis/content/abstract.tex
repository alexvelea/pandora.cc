The purpose of this paper is to introduce the reader to a new C++ framework called Pandora.cc while showcasing its potential uses and the reasoning behind various design choices. 

The framework is a blend of micro and nano services architecture. It provides fast and powerful parallelism capabilities for writing nanoservices using a standard RPC-server interface. The creation of bundles of nano services allows the creation of microservices that can be deployed in the cloud, allowing the code to scale without requiring code changes. Each Pandora.cc binary can be configured with ease to serve a subset of services along with technical hardware specifications regarding thread distribution, allowing lock-free code to emerge and to allocate more resources to latency-sensitive services.

The paper concludes with various benchmarks to present experimentally that the framework can be used successfully in both multithread and distributed systems. The presence of both code examples and UML diagrams showcase the slick and modern API design and functionality supported while the introduction of the various design principles helps users decide if Pandora.cc is a good fit for them.