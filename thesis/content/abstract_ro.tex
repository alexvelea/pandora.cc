Scopul lucr\u{a}rii e de a familiariza cititorul cu noua librarie de C++, Pandora.cc, \^{i}n acela\c{s}i timp, ar\u{a}t\^{a}nd potentiale utiliz\u{a}ri si explic\^{a}nd principiile de design care au stat la crearea acesteia.

Libr\u{a}ria combin\u{a} nuan\c{t}e din architecturile software de micro si nano servicii. Ofer\u{a} capabilit\u{a}ti rapide \c{s}i configurabile de papalelism pentru dezvoltarea applica\c{t}iilor folosind nanoserviciilor \^{i}mpreun\u{a} cu o interfa\c{t}\u{a} standard de tip RPC. Abilitatea de a crea grupuri de nanoservicii este folosit\u{a} pentru a definii componente similare cu microserviciile, ca mai apoi acestea s\u{a} fie rulate in cloud, permit\^{a}nd o scalabilitate orizontala f\u{a}r\u{a} a modifica codul existent. Fiecare executabil de tipul Pandora.cc poate fi configurat cu u\c{s}urin\c{t}\u{a} s\u{a} satisfac\u{a} un subset din serviciile implementate, fiind posibil\u{a} o configurare pentru a preciza ce resurse hardware sunt folosite de servicii, astfel obtind un cod f\u{a}r\u{a} mutexi care poate gestiona mai bine resursele de intrare/ie\c{s}ire.

Lucrarea se sfar\c{s}e\c{s}te cu diverse teste de performant\u{a} care atest\u{a} experimental c\u{a} libr\u{a}ria poate fi folosit\u{a} cu succes at\^{a}t pentru aplica\c{t}ii multithreaded c\^{a}t si ca o optiune de calcul distribuit. Prezen\c{t}a at\^{a}t a digramelor UML c\^{a}t si a exemplelor de cod atest\u{a} un API modern \c{s}i usor de folosit iar introducerea \^{i}n lucrare a principilor de design din spatele libr\u{a}riei permite utilizatorilor s\u{a} decid\u{a} dac\u{a} poate fi considerat\u{a} o optiune viabila pentru cazul personal. 